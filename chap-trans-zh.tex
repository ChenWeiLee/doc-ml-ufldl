%# -*- coding: utf-8 -*-
%!TEX encoding = UTF-8 Unicode
%!TEX TS-program = xelatex
% vim:ts=4:sw=4
%
% 以上设定默认使用 XeLaTex 编译,并指定 Unicode 编码,供 TeXShop 自动识别

% Author: Yunhui Fu <yhfudev@gmail.com>
% License: Creative Commons (CC BY 4.0)

\section{\cnt{Chinese Version}{中文版本说明}{}}


\subsection{\cnt{Chinese-English dictionary}{中英文词汇对照}{}}


\begin{longtable}[h]{m{0.4\textwidth}m{0.4\textwidth}}
\caption{中英文词汇对照} \label{tab:zhendict} \\
\toprule
\textbf{英文}&\textbf{中文} \\
\midrule
\endfirsthead  % endfirsthead 以上內容只出現在第一頁
\midrule
\textbf{英文}&\textbf{中文} \\
\midrule
\endhead % endfirsthead 及 endhead 間的內容會排版至每頁上方
%\midrule
\multicolumn{2}{r}{续下页 \dots} \\
\midrule

\endfoot % endfoot 以上的內會出現在每頁底部
\endlastfoot

neural networks & 神经网络 \\
  \midrule

activation function & 激活函数 \\
  \midrule

hyperbolic tangent & 双曲正切函数 \\
  \midrule

bias units & 偏置项 \\
  \midrule

activation & 激活值 \\
  \midrule

forward propagation & 前向传播 \\
  \midrule

feedforward neural network & 前馈神经网络(参照Mitchell的《机器学习》的翻译) \\
  \midrule

Backpropagation Algorithm & 反向传播算法 \\
  \midrule

(batch) gradient descent & (批量)梯度下降法 \\
  \midrule

(overall) cost function & (整体)代价函数 \\
  \midrule

squared-error & 方差 \\
  \midrule

average sum-of-squares error & 均方差 \\
  \midrule

regularization term & 规则化项 \\
  \midrule

weight decay & 权重衰减 \\
  \midrule

bias terms & 偏置项 \\
  \midrule

Bayesian regularization method & 贝叶斯规则化方法 \\
  \midrule

Gaussian prior & 高斯先验概率 \\
  \midrule

MAP & 极大后验估计 \\
  \midrule

maximum likelihood estimation & 极大似然估计 \\
  \midrule

activation function & 激活函数 \\
  \midrule

tanh function & 双曲正切函数 \\
  \midrule

non-convex function & 非凸函数 \\
  \midrule

hidden (layer) units & 隐藏层单元 \\
  \midrule

symmetry breaking & 对称失效 \\
  \midrule

learning rate & 学习速率 \\
  \midrule

forward pass & 前向传导\\
  \midrule

hypothesis & 假设值 \\
  \midrule

error term & 残差 \\
  \midrule

weighted average & 加权平均值 \\
  \midrule

feedforward pass & 前馈传导 \\
  \midrule

Hadamard product & 阿达马乘积 \\
  \midrule

forward propagation & 前向传播 \\
  \midrule

off-by-one error & 缺位错误 \\
  \midrule

bias term & 偏置项 \\
  \midrule

numerically checking & 数值检验 \\
  \midrule

numerical roundoff errors & 数值舍入误差 \\
  \midrule

significant digits & 有效数字 \\
  \midrule

unrolling & 组合扩展 \\
  \midrule

learning rate & 学习速率 \\
  \midrule

Hessian matrix & Hessian 矩阵 \\
  \midrule

Newton's method & 牛顿法 \\
  \midrule

conjugate gradient & 共轭梯度 \\
  \midrule

step-size & 步长值 \\
  \midrule

自编码算法 Autoencoders \\
  \midrule

稀疏性 Sparsity \\
  \midrule

神经网络 neural networks \\
  \midrule

监督学习 supervised learning \\
  \midrule

无监督学习 unsupervised learning \\
  \midrule

隐藏神经元 hidden units \\
  \midrule

像素灰度值 the pixel intensity value \\
  \midrule

独立同分布 IID \\
  \midrule

主元分析 PCA \\
  \midrule

激活 active \\
  \midrule

抑制 inactive \\
  \midrule

激活函数 activation function \\
  \midrule

激活度 activation \\
  \midrule

平均活跃度 the average activation \\
  \midrule

稀疏性参数 sparsity parameter \\
  \midrule

惩罚因子 penalty term \\
  \midrule

相对熵 KL divergence \\
  \midrule

伯努利随机变量 Bernoulli random variable \\
  \midrule

总体代价函数 overall cost function \\
  \midrule

后向传播 backpropagation \\
  \midrule

前向传播 forward pass \\
  \midrule

梯度下降 gradient descent \\
  \midrule

目标函数 the objective \\
  \midrule

梯度验证方法 the derivative checking method \\
  \midrule

可视化 Visualizing \\
  \midrule

自编码器 Autoencoder \\
  \midrule

隐藏单元 hidden unit \\
  \midrule

非线性特征 non-linear feature \\
  \midrule

激励 activate \\
  \midrule

平凡解 trivial answer \\
  \midrule

范数约束 norm constrained \\
  \midrule

稀疏自编码器 sparse autoencoder \\
  \midrule

有界范数 norm bounded \\
  \midrule

输入域 input domains \\
  \midrule

逻辑回归 Logistic Regression \\
  \midrule

批量梯度上升法 batch gradient ascent \\
  \midrule

截距 intercept term \\
  \midrule

对数似然函数 the log likelihood \\
  \midrule

导函数 derivative \\
  \midrule

梯度 gradient \\
  \midrule

向量化 vectorization \\
  \midrule

正向传播 forward propagation \\
  \midrule

反向传播 backpropagation \\
  \midrule

训练样本 training examples \\
  \midrule

激活函数 activation function \\
  \midrule

稀疏自编码网络 sparse autoencoder \\
  \midrule

稀疏惩罚 sparsity penalty \\
  \midrule

平均激活率 average firing rate \\
  \midrule

Principal Components Analysis 主成份分析 \\
  \midrule

whitening 白化 \\
  \midrule

intensity 亮度 \\
  \midrule

mean 平均值 \\
  \midrule

variance 方差 \\
  \midrule

covariance matrix 协方差矩阵 \\
  \midrule

basis 基 \\
  \midrule

magnitude 幅值 \\
  \midrule

stationarity 平稳性 \\
  \midrule

normalization 归一化 \\
  \midrule

eigenvector 特征向量 \\
  \midrule

eigenvalue 特征值 \\
  \midrule

白化 whitening \\
  \midrule

冗余 redundant \\
  \midrule

方差 variance \\
  \midrule

平滑 smoothing \\
  \midrule

降维 dimensionality reduction \\
  \midrule

正则化 regularization \\
  \midrule

反射矩阵 reflection matrix \\
  \midrule

去相关 decorrelation \\
  \midrule

Softmax回归 Softmax Regression \\
  \midrule

有监督学习 supervised learning \\
  \midrule

无监督学习 unsupervised learning \\
  \midrule

深度学习 deep learning \\
  \midrule

logistic回归 logistic regression \\
  \midrule

截距项 intercept term \\
  \midrule

二元分类 binary classification \\
  \midrule

类型标记 class labels \\
  \midrule

估值函数/估计值 hypothesis \\
  \midrule

代价函数 cost function \\
  \midrule

多元分类 multi-class classification \\
  \midrule

权重衰减 weight decay \\
  \midrule

自我学习/自学习 self-taught learning \\
  \midrule

无监督特征学习 unsupervised feature learning \\
  \midrule

自编码器 autoencoder \\
  \midrule

白化 whitening \\
  \midrule

激活量 activation \\
  \midrule

稀疏自编码器 sparse autoencoder \\
  \midrule

半监督学习 semi-supervised learning \\
  \midrule

自我学习 self-taught learning \\
  \midrule

深层网络 deep networks \\
  \midrule

微调 fine-tune \\
  \midrule

稀疏自编码器 sparse autoencoder \\
  \midrule

梯度下降 gradient descent \\
  \midrule

非监督特征学习 unsupervised feature learning \\
  \midrule

    pre-training & 预训练 \\
  \midrule

深度网络 Deep Networks \\
  \midrule

深度神经网络 deep neural networks \\
  \midrule

非线性变换 non-linear transformation \\
  \midrule

激活函数 activation function \\
  \midrule

简洁地表达 represent compactly \\
  \midrule

“部分-整体”的分解 part-whole decompositions \\
  \midrule

目标的部件 parts of objects \\
  \midrule

高度非凸的优化问题 highly non-convex optimization problem \\
  \midrule

共轭梯度 conjugate gradient \\
  \midrule

梯度的弥散 diffusion of gradients \\
  \midrule

逐层贪婪训练方法 Greedy layer-wise training \\
  \midrule

自动编码器 autoencoder \\
  \midrule

微调 fine-tuned \\
  \midrule

自学习方法 self-taught learning \\
  \midrule

栈式自编码神经网络(可以考虑翻译为“多层自动编码机”或“多层自动编码神经网络”) Stacked autoencoder \\
  \midrule

微调 Fine tuning \\
  \midrule

反向传播算法 Backpropagation Algorithm \\
  \midrule

前馈传递 feedforward pass \\
  \midrule

激活值 (可以考虑翻译为“激励响应”或“响应”) activation \\
  \midrule

线性解码器 Linear Decoders 

稀疏自编码 Sparse Autoencoder 

输入层 input layer 

隐含层 hidden layer 

输出层 output layer 

神经元 neuron 

神经网络 neural network 

自编码器 autoencoder 

激励函数 activation function 

鲁棒 robust 

S型激励函数 sigmoid activation function 

tanh激励函数 tanh function 

线性激励函数 linear activation function 

恒等激励函数 identity activation function 

隐单元 hidden unit 

权重 weight 

偏差项 error term 

反向传播算法 backpropagation


全联通网络 Full Connected Networks 

稀疏编码 Sparse Autoencoder 

前向输送 Feedforward 

反向传播 Backpropagation 

部分联通网络 Locally Connected Networks 

连接区域 Contiguous Groups 

视觉皮层 Visual Cortex 

卷积 Convolution 

固有特征 Stationary 

池化 Pool 

    稀疏编码 Sparse Coding 
    无监督学习 unsupervised method 
    超完备基 over-complete bases 
    主成分分析 PCA 
    稀疏性 sparsity 
    退化 degeneracy 
    代价函数 cost function 
    重构项 reconstruction term 
    稀疏惩罚项 sparsity penalty 
    范式 norm 
    生成模型 generative model 
    线性叠加 linear superposition 
    加性噪声 additive noise 
    特征基向量 basis feature vectors 
    经验分布函数 the empirical distribution 
    KL 散度 KL divergence 
    对数似然函数 the log-likelihood 
    高斯白噪音 Gaussian white noise 
    先验分布 the prior distribution 
    先验概率 prior probability 
    源特征 source features 
    能量函数 the energy function 
    正则化 regularized 
    最小二乘法 least squares 
    凸优化软件convex optimization software 
    共轭梯度法 conjugate gradient methods 
    二次约束 quadratic constraints 
    拉格朗日对偶函数 the Lagrange dual 
    前馈结构算法 feedforward architectures 
    稀疏编码 sparse coding 
    自编码 autoencoder 
    目标函数 objective function 
    稀疏代价 sparsity cost 
    反向传播 backpropagation 
    基于梯度的 gradient-based 
    非凸的 non-convex 
    权重衰变 weight decay 
    拓扑稀疏编码 topographic sparse coding 
    拓扑秩序 topographically ordered 
    平滑的一范数惩罚 smoothed L1 penalty 
    迷你块 mini-batches 
    收敛速度 the rate of convergence 
    梯度下降 gradient descent 
    局部最优解 local optima
    归一化 normalization 

    白化 whitening 

    直流分量 DC component 

    局部均值消减 local mean subtraction 

    消减归一化 sparse autoencoder 

    缩放 rescaling 

    逐样本均值消减 per-example mean subtraction 

    特征标准化 feature standardization 

    平稳 stationary 

    Mel倒频系数 MFCC 

    零均值化 zero-mean 

    低通滤波 low-pass filtering 

    基于重构的模型 reconstruction based models 

    自编码器 autoencoders 

    稀疏编码 sparse coding 

    受限Boltzman机 RBMs 

    k-均值 k-Means 

    长尾 long tail 

    损失函数 loss function 

    正交化 orthogonalization
    反向传导 backpropagation 
    稀疏编码 sparse coding 
    权重矩阵 weight matrix 
    目标函数 objective 
    平滑地形L1稀疏罚函数 Smoothed topographic L1 sparsity penalty 
    重建代价 reconstruction cost 
    稀疏自编码器 sparse autoencoder 
    梯度 gradient 
    神经网络 neural network 
    神经元 neuron 
    激励 activation 
    激励函数 activation function 
    独立成分分析 independent component analysis 
    单位激励函数 identity activation function 
    平方函数 square function 
    分组矩阵 grouping matrix 
    特征矩阵 feature matrix 
    独立成分分析 Independent Component Analysis 
    稀疏编码算法 Sparse coding 
    超完备基 Over-complete basis 
    标准正交基 Orthonormal basis 
    稀疏惩罚项 Sparsity penalty 
    梯度下降法 Gradient descent 
    白化 Whitened 
    不完备基 Under-complete basis 
    线搜索算法 Line-search algorithm 
    拓扑代价项 Topographic cost term 
\\

\bottomrule
\end{longtable}

\subsection{\cnt{Translator}{翻译人员}{}}


\begin{longtable}[h]{cm{0.8\textwidth}}
\caption{翻译人员} \label{tab:zhtranslator} \\
\toprule
\textbf{章节}&\textbf{翻译} \\
\midrule
\endfirsthead  % endfirsthead 以上內容只出現在第一頁
\midrule
\textbf{章节}&\textbf{翻译} \\
\midrule
\endhead % endfirsthead 及 endhead 間的內容會排版至每頁上方
%\midrule
\multicolumn{2}{r}{续下页 \dots} \\
\midrule

\endfoot % endfoot 以上的內會出現在每頁底部
\endlastfoot

\ref{chp:neuralnet} & { 孙逊(\mailurl{sunpaofu@foxmail.com}),林锋(\mailurl{xlfg@yeah.net}),刘鸿鹏飞(\mailurl{just.dark@foxmail.com}),许利杰(\mailurl{csxulijie@gmail.com}) } \\
  \midrule

\ref{chp:bkpropgationalg} & 王方(\mailurl{fangkey@gmail.com}),林锋(\mailurl{xlfg@yeah.net}),许利杰(\mailurl{csxulijie@gmail.com}) \\
  \midrule

\ref{chp:gradcheckingopt} & 袁晓丹(\mailurl{shadowwalker1991@gmail.com}),王方(\mailurl{fangkey@gmail.com}),林锋(\mailurl{xlfg@yeah.net}),许利杰(\mailurl{csxulijie@gmail.com}) \\
  \midrule

\ref{chp:autoencsparse} & 周韬(\mailurl{ztsailing@gmail.com}),葛燕儒(\mailurl{yrgehi@gmail.com}),林锋(\mailurl{xlfg@yeah.net}),余凯(\mailurl{kai.yu.cool@gmail.com})  \\


\bottomrule
\end{longtable}

